\documentclass[12pt]{article}
\usepackage{graphicx}
\usepackage[margin=1.0in]{geometry}

\def\given{\, | \,}

\begin{document}

\title{Homework \# 1, Due January 18, 2023}
\author{BIOS 617}
\date{Assigned on: January 4, 2023}

\maketitle

\begin{enumerate}
\setlength{\itemsep}{15pt}%
\setlength{\parskip}{15pt}%

\item Exercise in relationship between marginal and conditional means and variances.

Consider worker income from four departments: $C_1, C_2, C_3$, and $C_4$ with $3, 2, 4$, and $3$ employees respectively:

\begin{table}[!th]
\centering
\begin{tabular}{| c c c c |}
\hline
$C_1$ & $C_2$ & $C_3$ & $C_4$ \\ \hline
50 & 90 & 18 & 32 \\
60 & 105 & 23 & 10 \\
70 & - & 22 & 60 \\
- & - & 25 & - \\ \hline
\end{tabular}
\end{table}

	\begin{enumerate}
		\setlength{\itemsep}{15pt}%
		\setlength{\parskip}{15pt}%

		\item Find the mean $E(y)$ and variance $V(y)$ where $y$ is a {\bf single} individual's income sampled at random. {\bf [10 pt]}
		\item Fill in the following table:
		\begin{table}[!th]
		\centering
		\begin{tabular}{| c | c c c c |} \hline
		& $C_1$ & $C_2$ & $C_3$ & $C_4$ \\ \hline
		$E(X \given C)$ & - & - & - & - \\
		$V(X \given C)$ & - & - & - & - \\
		$P(C)$ & - & - & - & - \\ \hline
		\end{tabular}
		\end{table}

		where $E ( X \given C )$ is the mean income in each department, $V ( X \given C )$ is the variance of the income in each department, and $P ( C )$ is the probability that an individual selected at random from the population will belong to that department. {\bf [10 pt]}

		\item Show numerically using the above example that $E (X) = E(E(X \given C) )$ and $V(X) = V(E(X \given C)) + E(V(X \given C))$ {\bf [10 pt]}

		\end{enumerate}

\item Suppose a population has mean~$\bar Y = 3$. Suppose two samples are acquired via simple random sampling with means~$\bar Y_1 = 1.5$ and~$\bar Y_2 = 4$ respectively.  Suppose we can only select a {\bf single unit} from either sample 1 or sample 2.  With probability $p_1$, we select a unit {\bf at random} from sample 1.  With probability~$p_2$ we select a unit {\bf at random} from sample 2.

	\begin{enumerate}
		\item Conditional on $\bar Y_i = \sum_{j=1}^{N_i} Y_j^{(i)}$ for $i=1,2$ and $\bar Y$, determine the selection probability $p_1$ associated with the first sample and the selection probability $p_2 = 1-p_1$ associated with the second sample required to so that the randomly chosen unit is an unbiased estimate of $\bar Y$.
		\item Suppose $N_1, N_2$ is the size of samples $1$ and $2$ respectively.  Define the variance for each sample
			$$S_i^2= \frac{1}{N_i-1} \sum_{j=1}^{N_i} (Y_j^{(i)} - \bar Y_i)^2 $$
		Determine the variance of $y$ for the sample design given by (a) in terms of $\bar Y_i, S_i^2, N_i$ for $i=1,2$ and $\bar Y$. Let $N_1 = N_2 = 100$ and $S_1 = S_2 = 2$.  Plug in to the derived formula to estimate the variance of $y$. {\bf [10 pts]}
	\end{enumerate}

\item Consider a small population of $N = 8$ students with the following exam grades: $Y_1 = 47$, $Y_2 = 66$, $Y_3 = 71$, $Y_4 = 75$, $Y_5 = 79$, $Y_6 = 82$, $Y_7 = 85$, $Y_8 = 90$.

	\begin{enumerate}
		\setlength{\itemsep}{15pt}%
		\setlength{\parskip}{15pt}%

		\item Compute the population mean $\bar Y$ and variance $S^2$. {\bf [5 pt]}
		\item Identify all simple random samples (sampling without replacement) of size $n=2$, compute their mean $(y_1 + y_2)/2$. {\bf [5 pt]}
		\item For the samples obtained in (b), show that the sample variances of $\bar y$ are given by $\frac{3}{16} (y_1 - y_2)^2$, and make a histogram of them. {\bf [5 pt]}
		\item Using the results of b) and c), show numerically that $E(\bar y) = \bar Y$ and $E (s^2) = S^2$. {\bf [10 pt]}
		\item Determine numerically whether $s$ is an unbiased estimator of $S$. {\bf [5 pt]}
	\end{enumerate}

\item Returning to the population in \# 3, consider a stratified sampling in which the population is stratified into low score ($<= 75$) and high score ($> 75$) strata.
	\begin{enumerate}
		\item Identify all possible random samples of size 4 obtained by selecting $n_h = 2$ (without replacement) from each stratum and make a histogram of the resulting means. {\bf [10 pt] }
		\item Compute the sampling variance of the sample means in (a) directly, and compare this to the sampling variance given by $\sum_{h=1}^2 P_h^2 \frac{1-f_h}{n_h} S_h^2$. {\bf [5 pt] }
		\item Compute the design effect of this sample design. {\bf [5 pt]}
	\end{enumerate}

\end{enumerate}
\end{document}